
\documentclass[a4paper]{article}

\usepackage[general]{fancymath}
\usepackage{fancycom}

\aname{Adam Walsh}
\stitle{D-modules discussion}


\begin{document}
\maketitle
Since $X$ is a smooth algebraic variety, that is the stalks are regular local rings (whatever that means), the tangent sheaf is locally free of rank $n$, i.e. for every affine open it is isomorphic to an $n$-direct sum of the structure sheaf.

The tangent sheaf consists of $\bC_X$-endomorphisms of $\cO_X$, which are presumable defined by locally being endomorphisms of $\bC$-algebras. These endomorphisms satisfy the Leibniz rule, the generalised product rule.

Sections of the tangent sheaf, i.e. vector fields on $X$ have by definition an action on $X$.

Sections of the structure sheaf act on other sections via left multiplication.

We thus have two sheaves of endomorphisms on $\cO_X$, and we may use these sheaves to generate the sheaf $D_X$ of differential operators on $X$.

In the case of a manifold, given a point $p$, we can take a chart having as its origin $p$, with coordinate functions $x_1,\dots,x_n$. We may then construct a basis of the tangent space by the tangents to the axis curves, denoted $\partial_i$.

What is this local coordinate system? We simultaneously defined bot ``coordinates'' on the affine open $U$, and the ``ta

For the manifold $M$, the coordinates $x_i$ are precisely the coordinate projection functions taking $(a_1,\dots,a_n)$ to $a_i$. The $\partial_i$ are immediately then induced by treating the inverse of the coordinate function as a curve, and taking the tangent vector.

An algebraic variety is smooth if all the stalks are \emph{regular local rings}, i.e. they are Noetherian local rings such that the minimal number of generators of the maximal ideal is equal to the krull dimension.

Recall that a Noetherian ring is such that every ascending chain of prime ideals terminates, which defines the Krull dimension in the first place.

Essentially, $D_U$ is formed of all polynomials in the variables $\partial_i$.

Note that multiplication of polyomials in $\partial_i$ and in $x_i$ will produce a polynomial in $x_i$ times a polynomials in $\partial_i$.


\section{Talk Structure}
Contents\\ 5
Motivation for $D$-modules\\ 5
Concepts in the land of DG\\ 5
Idea of vector fields in DG\\ 5
Explanation of why vector fields act on sections\\ 5
Explanation of algebraic variety\\ 5
Examples of algebraic variety\\ 5
Note on Smooth algebraic variety\\ 5
Define derivation\\ 5
Introduce Tangent sheaf\\ 5
Local coordinate system in DG\\ 2
Local coordinate system in Sheaf\\ 3
Local definition of tangent sheaf from locally free, with basis\\ 5
Local definition of sheaf of differential operators\\ 5
Sheaf of differential operators as $\bC$ algebra satisfying relations\\ 5
Total symbol, proof of Leibniz rule?\\ 10
Definition of order filtration?\\ 10
Definition of sheaf of differential operators in terms of order filtration?\\ 10
Graded rings?\\ 20
principal symbol?\\
\end{document}
