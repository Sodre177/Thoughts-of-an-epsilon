\documentclass[a4paper]{article}

\usepackage[general]{fancymath}
\usepackage{fancycom}

\aname{Adam Walsh}
\stitle{A small note on Algebraic varieties}

\begin{document}
\maketitle
Consider an affine (i.e. finitely generated and reduced) $k$-algebra $A$. We define the maximal spectrum $V = \Specm(A)$ in the usual way, as the set of maximal ideals of $A$ endowed with the \emph{Zariski} topology. A closed set in the Zariski topology is the set of maximal ideals containing some some fixed collection of functions $I \subseteq A$. Note that the collection of functions may as well be replaced by the ideal they generate.
We may also define the Zariski topology by a basis of open sets known as the \emph{distinguished opens}, given by
\[
    D(f) \ceq \{\fkm_x \in V \st f \nin \fkm_x\} \cndall f \in A.
\]

We may consider $A$ as a collection of $k$-valued functions taking inputs from $V$, by defining
\[
    f(x) = f(\fkm_x) = f + \fkm_x \in A/\fkm_x.
\]
That is, the surjective algebra homomorphism
\[
    \vfunc{e_x}{A}{A/\fkm_x}{f}{f + \fkm_x}
\]
forms the evaluation map for functions in $A$ at points of $V$.

We may now define sheaf structure $\cO_V$ on $V$ by associating to each open set $D(f)$ the ring
\[
    \Gamma(D(f)) \ceq A_f \ceq S^{-1}A, \qquad S = \{1, f, f^2, \dots\}.
\]
The corresponding stalks are easily seen to be
\[
    \cO_{V,x} = A_x = (A\bsl\fkm_x)^{-1}A.
\]
An affine variety is then a ringed space isomorphic to such a spectrum of a ring with associated structure sheaf.

An algebraic pre-variety is a sheaf of rings, with sections considered as $k$-valued functions, that is locally isomorphic to an affine variety.

An algebraic variety is a quasi-compact, separated pre-variety.
\end{document}
