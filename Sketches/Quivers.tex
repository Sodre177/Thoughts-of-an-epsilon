\documentclass[a4paper]{amsart}

\usepackage{fancycom}

\title{Quiver representations}
\author{Adam Walsh}

\begin{document}
\maketitle
\begin{abstract}
    In this sketch we examine some basic properties of quivers and quiver representations from an algebraic point of view. We shall follow the notation of [1].
\end{abstract}
Quivers are simple mathematical objects which provide easy descriptions of many concepts, particularly in representation theory. We begin by giving the definition of a quiver, then we develop the category of representations associated to each quiver, and conclude with a brief look at the quiver algebra.
\section*{Quivers}
\begin{Definition}
    A quiver is a graph with directed edges where multiple edges are allowed between vertices. Formally, a quiver $Q=(Q_0,Q_1,s,t)$ consists of four pieces of information:
    \begin{itemize}
        \item The set of vertices $Q_0$;
        \item The set of edges $Q_1$; and
        \item Functions $\func{s,t}{Q_1}{Q_0}$ sending an edge to its source and target, respectively.
    \end{itemize}
\end{Definition}
For simplicity, unless otherwise specified all quivers in this document will be finite and connected.

We usually denote the vertices of $Q$ with natural numbers $i$ and the edges by $\alpha_j$. For example, the following depicts a quiver with three vertices $Q_0=\{1,2,3\}$ and edges $Q_1=\{\alpha_1,\alpha_2\}$ going from $1$ to $2$ and $3$ to $2$ respectively.
\[
    \begin{tikzcd}
        1 \arrow[r, "\alpha_1"] & 2 & 3 \arrow[l,swap, "\alpha_2"]
    \end{tikzcd}
\]
Another example of a quiver is the $r$-arrow Kronecker quiver, denoted by $K_r$. $K_r$ consists of two vertices, usually denoted $1$ and $2$, along with $r$ arrows $\alpha_1,\dots,\alpha_r$ from $1$ to $2$.
\[
    \begin{tikzcd}
        1
            \arrow[r, draw=none, "\raisebox{+1.5ex}{\vdots}" description]
            \arrow[r, bend left,        "\alpha_1"]
            \arrow[r, bend right, swap, "\alpha_r"]
        &  2
    \end{tikzcd}
\]
We are now ready to define quiver representations. Fix a base field $k$. A representation of a quiver assigns to each vertex a $k$-vector space and to each arrow a $k$-linear map between the corresponding vector spaces.
%We can now define quiver representations. Fix a base field $k$.
\begin{Definition}
    Let $Q$ be a quiver with vertices labelled $i$ and edges $\alpha_j$. Then a quiver representation $M = (M_i, \varphi_j)$ consists of:
    \begin{itemize}
        \item $k$-vector spaces $M_i$ corresponding to each vertex $i$; and
        \item $k$-linear maps $\func{\varphi_j}{M_{s(\alpha_j)}}{M_{t(\alpha_j)}}$ associated to each edge $\alpha_j$.
    \end{itemize}
\end{Definition}
To write specific quiver representations, we recall that a $k$-linear map from $k^m$ to $k^n$ may be written as an $n\times m$ matrix. For example, the following is a representation of $K_2$.
\[
    \begin{tikzcd}
        k^1
            \arrow[r, bend left,        "\sbmqty{1\\0}"]
            \arrow[r, bend right, swap, "\sbmqty{0\\1}"]
        &  k^2
    \end{tikzcd}
\]

\end{document}

