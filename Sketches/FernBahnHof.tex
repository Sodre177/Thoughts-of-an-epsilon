\documentclass[a4paper]{article}

\usepackage[general]{../fancymath}
\usepackage{../fancycom}

\aname{Adam Walsh}
\stitle{A proof of the FernBahnHof Theorem}

%\rawtitle{ }
%\rawauthor{ }
%\rawrfoot{ }
%\rawlfoot{ }
%\rawlhead{ }

\def\into{\hookrightarrow}
\def\mono{\rightarrowtail}
\def\epi{\twoheadrightarrow}


\begin{document}
\maketitle
\section{Introduction}
In this note we prove the FernBahnHof theorem from (reference to Vakil exercise here).
%Reference theorem in later section
%Explain why it is interesting
%Give the general outline
\section{Basic concepts}
Before stating the theorem, we must first define some required terminology.\\

%Define Abelian Category, embedding
%In abelian cats the image is a monomorphism
%Short exact sequences, relation to kernel, cokernel and quotient
%Several methods of factoring in AbCat
%Additive functors
%Define left, right, and exact functors
%Define chain complexes
\begin{Definition}
    An \emph{ascending} chain complex $(C^*, f^*)$ is a diagram
    \[
        \dots \stackrel{f^{i-2}}{\to} C^{i-1} \stackrel{f^{i-1}}{\to} C^{i} \stackrel{f^{i}}{\to} C^{i+1} \stackrel{f^{i+1}}{\to} \dots
    \]
    such that $f^{i} \circ f^{i-1} = 0$ for all $i$.
\end{Definition}
Since we are working in an abelian category, each connecting map $f^i$ uniquely induces the short exact sequence:
\[
    \Ker(f^i) \mono C^{i} \epi \im(f^i).
\]
Further, the condition that $f^i \circ f^{i-1} = 0$ precisely says that the image of $f^{i-1}$ is a subobject of the kernel of $f^i$, hence we have a monomorphism
\[ %This needs a proof
    \im(f^{i-1}) \into \Ker(f^{i}).
\]
We define the \emph{cohomology} at $C^i$ to be the associated quotient $H^i(C^*) \ceq \im(f^{i-1})/\Ker(f^i)$.\\

For each $i$, the cohomology induces the sequence
\[ %Also needs proof
    H^i(C^*) \mono \CoKer(f^i) \epi \im(f^i).
\]

\section{The FHHF Theorem}
%Reason for name?
%Lemma for image map
%Theorem
\begin{Theorem}[The FernbaHnHoF Theorem]
    Let $\func{F}{\cA}{\cB}$ be an additive covariant functor of abelian categories, and consider an ascending complex $\big(C^{*}, f^*\big)$ in $\cA$.
    \begin{enumerate}
        \item If $F$ is a right exact functor, there is a natural family of morphisms $\func{\phi^i}{FH^{i}(C^{*})}{H^{i}(FC^{*})}$ in $\cB$.
        \item Similarly, if $F$ is a left exact functor, there is a natural family of morphisms $\func{\varphi^i}{H^{i}(FC^{*})}{FH^{i}(C^{*})}$ in $\cB$. 
        \item If $F$ is an exact functor, then the morphisms $\phi^i, \varphi^i$ above are inverses and hence form isomorphisms ${FH^{i}(C^{*})} \isom {H^{i}(FC^{*})}$.
    \end{enumerate}
\end{Theorem}
Before proving this theorem, we prove the following helpful Lemma:
\begin{Lemma}
    For $\func{F}{\cA}{\cB}$ a covariant additive right exact functor of abelian categories, given any $\phi \in \Hom_{\cA}(A_1,A_2)$, there is a natural epimorphism %Wth does natural mean here.. makes a diagram commute?
    \[
        F\big(\Im(\phi)\big) \epi \Im\big(F(\phi)\big).
    \]
\end{Lemma}
\begin{proof}
    Write the kernel/image factorisation of $\phi$ as
    \[
        \Ker(\phi) \stackrel{\iota}{\mono} A_1 \stackrel{\rho}{\epi} \Im(\phi).
    \]

\end{proof}
We can now prove the main theorem:
\begin{proof}
    .
\end{proof}
%Proof of theorem
\section{Examples}
%Something to do with tensors, Tor, and flatness.
\end{document}

