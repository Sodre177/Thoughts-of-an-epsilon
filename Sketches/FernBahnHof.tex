\documentclass[a4paper]{article}

\usepackage[general]{fancymath}
\usepackage{fancycom}

\aname{Adam Walsh}
\stitle{A proof of the FernBahnHof Theorem}

%\rawtitle{ }
%\rawauthor{ }
%\rawrfoot{ }
%\rawlfoot{ }
%\rawlhead{ }

\def\into{\hookrightarrow}
\def\mono{\rightarrowtail}
\def\epi{\twoheadrightarrow}


\begin{document}
\maketitle
In this note we give a proof of the FernBahnHof (henceforth FHHF) theorem, which details how the cohomology functor interacts with left, right and fully exact functors of abelian categories. (References and outline and stuff)
%Reference theorem in later section
%Explain why it is interesting
%Give the general outline
\section{Preliminary notions}
We begin by recalling some definitions from category theory required for the statement of the theorem.\\
In any category $\cA$ with a zero object $0$, the \emph{kernel} $(K,i)$ and \emph{cokernel} $(C,\rho)$, if they exist, of a morphism $\phi \in \Hom_\cA(A,B)$, are respectively defined by the following universal properties.
\[
    \begin{tikzcd}
        K \arrow[r, "i", hook] \arrow[rr, "0"', bend left]& A \arrow[r, "\phi"] & B \\
        & K' \arrow[lu, "\exists!", dashed] \arrow[u] \arrow[ur, "0"] &
    \end{tikzcd}
    \qquad
    \begin{tikzcd}
        A \arrow[r, "\phi"] \arrow[rr, "0"', bend left] \arrow[rd, "0"]& B \arrow[d] \arrow[r, "\rho", two heads] & C \arrow[ld, "\exists!", dashed]\\
        & C' &
    \end{tikzcd}
\]
The \emph{image} of $\phi$, denoted $\im(\phi)$ is the kernel of the cokernel, and the \emph{coimage} of $\phi$, denoted $\CoIm(\phi)$, is the cokernel of the kernel.\\


An \emph{abelian category} is, informally, a generalisation of the category of modules over a ring. It is an additive category where every morphism has a kernel and cokernel such that the usual first isomorphism theorem holds. More formally, we have the following definition.
\begin{Definition}
    An additive category $\cA$ with all kernels and cokernels (sometimes called a pre-abelian category) is called an abelian category if either of the following equivalent conditions hold.
    \begin{enumerate}
        \item All epimorphisms are cokernels and all monomorphisms are kernels.
        \item For any morphism $\phi\in \Hom_{\cA}(A,B)$, the central map $\hat{\phi}$ of the canonical factorisation
            \[
                A \epi \CoIm(\phi) \stackrel{\hat{\phi}}{\to} \im(\phi) \mono B
            \]
            is an isomorphism.
    \end{enumerate}
\end{Definition}
%We shall require several properties of abelian categories, which we now state without proof.
Henceforth all categories in this note are abelian.\\


A sequence of objects and morphisms
\[
    \dots \stackrel{f^{i-2}}{\to} C^{i-1} \stackrel{f^{i-1}}{\to} C^{i} \stackrel{f^{i}}{\to} C^{i+1} \stackrel{f^{i+1}}{\to} \dots
\]
is said to be a \emph{complex} if $\im(f^{i-1})$ is a subobject of $\Ker(f^i)$, or equivalently if $f^{i} \circ f^{i-1} = 0$, for all $i$. We often use the notation $(C^*, f^*)$ or simply $C^*$ to denote such a sequence. In particular, if $\im(f^{i-1}) = \Ker(f^i)$ holds for every $i$, we say that $C^*$ is \emph{exact}.

An exact sequence of the form
\[
    0 \to A \stackrel{f}{\mono} B \stackrel{g}{\epi} C \to 0
\]
is called a \emph{short exact sequence}. We often just write $A \stackrel{f}{\mono} B \stackrel{g}{\epi} C$, where the arrows denote a monomorphism and an epimorphism respectively.\\

Let $B$ be a (representative of a) subobject of $A$. Then the \emph{quotient} $A/B$ is defined as the cokernel of the monomorphism $B\into A$. Saying that a sequence $A \stackrel{f}{\mono} B \stackrel{g}{\epi} C$ is short exact may be interpreted as saying that $A$ is a subobject of $B$, and that $C$ is isomorphic to the quotient $B/A$, or equivalently that $(C,g)$ is the cokernel of $f$ and $(A,f)$ is the cokernel of $g$.\\

%We now define three types of functors which preserve part or all of short exact sequences.
\begin{Definition}
    Let $\func{F}{\cA}{\cB}$ be an additive covariant functor of abelian categories.
    \begin{enumerate}
        \item We call $F$ a \emph{left exact functor} if all exact sequences of the form $A \stackrel{f}{\mono} B \to C$ in $\cA$ get mapped to exact sequences $F(A) \stackrel{F(f)}{\mono} F(B) \to F(C)$.
        \item Similarly, we call $F$ a \emph{right exact functor} if exact sequences of the form $A \to B \stackrel{f}{\epi} C$ get mapped to exact sequences $F(A) \to F(B) \stackrel{F(f)}{\epi} F(C)$.
        \item Finally, we call $F$ an \emph{exact functor} if it is both left and right exact.
    \end{enumerate}
\end{Definition}
Note that left exact functors preserve kernels and in particular monomorphisms, and right exact functors preserve cokernels and epimorphisms. Fully exact functors will preserve both, as well as exactness in general.\\

Fix some complex $(C^*, f^*)$. Then the \emph{cohomology} at $C^i$, $H^{i}(C^*) \ceq \Ker(f^i)/\im(f^{i-1})$, is the quotient associated to the monomorphism $\im(f^{i-1}) \into \Ker(f^{i})$.\\

Finally, we give a construction of cohomology in terms of short exact sequences.
\begin{Theorem}
    Given a complex $(C^*, f^*)$, for each $i$ there are short exact sequences
    \[
        0 \to \Ker(f^i) \mono C^{i} \epi \im(f^i) \to 0
    \]
    \begin{equation}\label{eqn:CoHomExact}
        0 \to H^i(C^*) \mono \CoKer(f^{i-1}) \epi \im(f^{i}) \to 0
    \end{equation}
\end{Theorem}
%Faaaark this is a long and painful process
\begin{proof}
    The first sequence is an immediate consequence of working in an abelian category. Consider $f^i \in \Hom(C^{i}, C^{i+1})$, then from the second definition of an abelian category, we obtain an isomorphism $\CoKer(\Ker(f^i)) \isom \Ker(\CoKer(f^i))$. Hence the image of $f^i$ is the cokernel of the kernel of $f^i$, and the sequence follows from the definition of cokernel.\\
    The second sequence (Incomplete).
\end{proof}
%Several methods of factoring in AbCat
\section{The FHHF Theorem}
%Reason for name?
%Lemma for image map
%Theorem
\begin{Theorem}[The FernbaHnHoF Theorem]
    Let $\func{F}{\cA}{\cB}$ be an additive covariant functor of abelian categories, and consider an ascending complex $\big(C^{*}, f^*\big)$ in $\cA$.
    \begin{enumerate}
        \item If $F$ is a right exact functor, there is a natural family of morphisms $\func{\phi^i}{FH^{i}(C^{*})}{H^{i}(FC^{*})}$ in $\cB$.
        \item Similarly, if $F$ is a left exact functor, there is a natural family of morphisms $\func{\varphi^i}{H^{i}(FC^{*})}{FH^{i}(C^{*})}$ in $\cB$. 
        \item If $F$ is an exact functor, then the morphisms $\phi^i, \varphi^i$ above are inverses and hence form isomorphisms ${FH^{i}(C^{*})} \isom {H^{i}(FC^{*})}$.
    \end{enumerate}
\end{Theorem}
To prove this theorem, we shall use the following result.
\begin{Lemma}
    For $\func{F}{\cA}{\cB}$ a covariant additive right exact functor of abelian categories, given any $\phi \in \Hom_{\cA}(A_1,A_2)$, there is an epimorphism %Wth does natural mean here.. makes what diagram commute?
    \[
        F\big(\im(\phi)\big) \epi \im\big(F(\phi)\big).
    \]
\end{Lemma}
\begin{proof}
    The image of a morphism is defined as the kernel of its cokernel, which in an abelian category is isomorphic to the cokernel of the kernel. We shall establish how $F$ acts on the kernel, then use the universal property of cokernels to tell us how $F$ acts on the image.\\
    The kernel of $\phi$ is given by
    \[
        \begin{tikzcd}
            \Ker(\phi) \arrow[r, "i", hook] \arrow[rr, "0", bend right] & A \arrow[r, "\phi"] & B
        \end{tikzcd}
    \]
    Recalling that additive functors preserve $0$, we apply $F$ to the above diagram and then use the definition of the kernel of $F(\phi)$ to obtain the following.
    \[
        \begin{tikzcd}
            \Ker(F(\phi)) \arrow[r, "i'", hook] \arrow[rr, "0"', bend left] & F(A) \arrow[r, "F(\phi)"] & F(B) \\
                & F\big(\Ker(\phi)\big) \arrow[ul, "\exists ! f", dashed] \arrow[ur, "0"] \arrow[u, "F(i)"]&
        \end{tikzcd}
    \]
    To introduce the desired images, we write the kernel/image factorisation of $\phi$ as follows.
    \[
        \Ker(\phi) \stackrel{i}{\into} A_1 \stackrel{\rho}{\epi} \im(\phi).
    \]
    Then, by the definition of the cokernel of $F(i)$, applying $F$ to the above diagram we obtain
    \[
        \begin{tikzcd}
            F\big( \Ker(\phi) \big) \arrow[r, "f"] \arrow[rr, "F(i)"', bend right] & \Ker\big( F(\phi) \big) \arrow[r, "i'", hook] & F(A_1) \arrow[r, "\rho'", two heads] \arrow[rd, "F(\rho)", two heads] & \im\big(F(\phi)\big)\,  \Big(= \CoKer(i') \Big)\\
            & & & F\big(\im(\phi)\big)\, \Big( = \CoKer\big( F(i) \big)\Big) \arrow[u, "\exists! \psi", dashed]
        \end{tikzcd}
    \]
    Since $\rho'$ and $F(\rho)$ are epi, $\psi$ must be epi, and we have our desired map $\func{\psi}{F\big(\im(\phi)\big)}{\im\big(F(\phi)\big)}$.
\end{proof}
We now finally have the tools to prove the theorem.
\begin{proof}
    \begin{enumerate}
        \item Let $F$ be a right exact functor, and fix $i$. Then from \ref{eqn:CoHomExact} we can write
        \[
            0 \to H^i(C^*) \stackrel{i}{\mono} \CoKer(f^{i-1}) \stackrel{\rho}{\epi} \im(f^i) \to 0
        \]
        Applying $F$, we have
        \[
             F(H^i(C^*)) \stackrel{F(i)}{\to} F\big(\CoKer(f^{i-1})\big) \stackrel{F(\rho)}{\epi} F\big(\im(f^i)\big) \to 0
        \]
        Again from \ref{eqn:CoHomExact}, we obtain 
        \[
            0 \to H^i\big(F(C^*)\big) \stackrel{i'}{\mono} \CoKer\big(F(f^{i-1})\big) \stackrel{\rho'}{\epi} \im\big(F(f^i)\big) \to 0
        \]
        Putting these together with the lemma, and recalling that right exact functors commute with cokernels, we have
        \[
            \begin{tikzcd}
                F\big( H^i(C^*)\big) \arrow[r,"F(i)"] & F\big(\CoKer(f^{i-1})\big) \arrow[d, equal]\arrow[r, "F(\rho)", two heads] & F\big(\im(f^i)\big) \arrow[d, two heads]\\
                H^i\big(F(C^*)\big) \arrow[r, "i'", hook] & \CoKer\big(F(f^{i-1})\big) \arrow[r, "\rho'", two heads] & \im\big(F(f^i)\big)
            \end{tikzcd}
        \]
        The universal property of the kernel of $\rho'$ gives us our desired map. (Need to show it commutes)
        \item
        \item
    \end{enumerate}
\end{proof}
%Proof of theorem
\section{Examples}
%Something to do with tensors, Tor, and flatness.
\end{document}

