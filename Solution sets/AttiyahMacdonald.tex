\documentclass[a4paper]{article}

\usepackage[general]{fancymath}
\usepackage{fancycom}

\aname{Adam Walsh}
\stitle{Commutative algebra problem solutions}


\begin{document}
\maketitle
These are my answers to the exercise problems given in Attiyah and Macdonald's Introduction to Commutative Algebra text.
\begin{qanda}
  \question
    Let $x$ be a nilpotent element of a ring $A$. Show that $1+x$ is a unit of $A$. Deduce that the sum of a nilpotent element and a unit is a unit.
  \answer
    Let $x$ be nilpotent in $A$, then $x$ is in the nilradical and hence the Jacobson radical of $A$. By the equivalent definition of the Jacobson radical, $1-xy$ is a unit for all $y\in A$, hence choosing $y = -1$ we see that $1+x$ is a unit.
    
    Now consider nilpotent $n$ in $A$ and unit $u$ with inverse $u^{-1}$. Since the nilradical is an ideal, $nu^{-1}$ is nilpotent, and so by the first part $(n+u)u^{-1} = nu^{-1} + 1$ is a unit. Hence $n+u$ is a unit, proving the desired result.
  \question
    Let $A$ be a ring and let $A[x]$ be the ring of polynomials in an indeterminate $x$, with coefficients in $A$. Let $f = a_0 + a_1 x + \dots + a_n x^n\in A[x]$. Prove that
    \begin{enumerate}
        \item $f$ is a unit in $A[x]$ $\iff$ $a_0$ is a unit in $A$ and $a_1, \dots, a_n$ are nilpotent. [If $b_0 + b_1x + \dots + b_m x^m$ is the inverse of $f$, prove by induction on $r$ that $a_n^{r+1}b_{m-r} = 0$. Hence show that $a_n$ is nilpotent, and then use Ex. 1.]
        \item $f$ is nilpotent $\iff$ $a_0, a_1, \dots, a_n$ are nilpotent.
        \item $f$ is a zero-divisor $\iff$ there exists $a\neq 0$ in $A$ such that $af = 0$. [Choose a polynomial $g = b_0 + b_1 x + \dots + b_m x^m$ of least degree $m$ such that $fg = 0$. Then $a_n b_m = 0$, hence $a_n g = 0$ (because $a_ng$ annihilates $f$ and has degree $< m$). Now show by induction that $a_{n-r} g = 0$ $(0 \leq r \leq n)$.]
        \item $f$ is said to be \emph{primitive} if $(a_0,a_1, \dots, a_n) = (1)$. Prove that if $f,g \in A[x]$, then $fg$ is primitive $\iff$ $f$ and $g$ are primitive.
    \end{enumerate}
  \answer
    \emph{Postponed}
  \question
    Generalise the results of Exercise 2 to a polynomial ring $A[x_1,\dots,x_r]$ in several indeterminates.
  \answer
    \emph{Postponed}
  \question 
    In the ring $A[x]$, the Jacobson radical is equal to the nilradical.
  \answer
      Consider any $f = a_0 + a_1 x + \dots + a_n x^n$ in the Jacobson radical of $A[x]$. By the equivalent definition, $1 + fg$ is a unit for all $g \in A[x]$. Consider $g=x$, then by Ex. 2, for $1+fx = 1 + a_0 x + a_1 x^2 + \dots + a_n x^{n+1}$ to be a unit in $A[x]$ we must have $a_0,\dots,a_n$ nilpotent. Again by Ex. 2, this implies that $f$ is nilpotent and hence in the nilradical. As the nilradical is always contained in the Jacobson ideal, we have shown that the Jacobson ideal is equal to the nilradical.
\end{qanda}

%We want to show that $fg$ is primative if and only if $f$ and $g$ are primitive.

%Let $$f=\sum_{i=0}^n a_ix^i,\quad g=\sum_{i=0}^k b_ix^i, (\text{ wlog }) k\leq n,$$
%then extend the index for $g$, where $b_i=0$ for $n\geq i>k$.

%Then $fg=\sum_{i,j=0}^n a_ib_jx^{i+j} = a_nb_n x^{2n} + (a_{n-1}b_n + a_nb_{n-1}) x^{2n-1} + \dots + a_1b_1$.

%Assume that $fg$ is primative, so that $(\sum_{i+j=k}a_ib_j)_{k=0}^{2n}=(1)$. This ideal is contained in $(a_i)(b_j)$ and hence $(a_i)(b_j)=(1)$. $(a_i)(b_j)\subset (a_i)\cap (b_j)$ and hence $(a_i)\cap (b_j)=(1)$ and these are primative.

%Conversely $(a_i)=(1)=(b_j)$, in which case $(a_i)(b_j)=(a_ib_j)_{i,j=0}^n=(1)$.
\end{document}

