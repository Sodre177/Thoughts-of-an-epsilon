\documentclass[a4paper]{article}

\usepackage[general]{fancymath}
\usepackage{fancycom}

\aname{Adam Walsh}
\stitle{D-modules Talk notes}

\def\Diff{\operatorname{Diff}}

\begin{document}
\maketitle
\begin{itemize}
    \item We aim to define $D$-modules in a sheaf theoretic sense
    \item Setting, smooth algebraic variety $(X, \cO_X)$
    \item Algebraic variety over $\bC$ is irreducible top space $X$, locally ringed space of $\bC$-algebras, sections $\bC$ valued functions.
    \item Locally looks like affine algebraic variety over $\bC$
    \item Affine algebraic variety over $\bC$ is $(Y=mSpec(A), \cO_Y)$ for
    \item $A$ is an \emph{affine $\bC$-algebra}, i.e. finitely generated and reduced.
    \item Looks like polynomial ring quotient by some ideal.
    \item Smooth gives us we can use the same number of variables in each affine open.
    \item Now revising differentials on a $\bC$-algebra. Think of the $\bC$-algebra as the affine algebra.
    \item Consider $\bC$-algebra $A$, endomorphism ring on $A$.
    \item Endomorphs have Lie algebra structure given by commutator
    \item Derivations are endos which satisfy the Leibniz rule:
        \[
            \Der_\bC(A) \ceq \{\theta \in \End_\bC(A) \st \theta(fg) = \theta(f)g - f\theta(g) \cndall f,g\in A\}.
        \]
    \item Can identify elements of $A$ with endomorphisms by
        \[
            f \mapsto f\cdot -
        \]
    \item Nice way of acting derivations on elements of the algebra using the Lie bracket, since
        \[
            [\theta, f](g) = \theta \cdot f(g) - f \cdot \theta(g) = \theta(fg) - f \cdot \theta(g) = (\theta(f))g.
        \]
    \item Since $[\theta, f]$ is a section, $[ [\theta, f], g] = 0$. 
    \item The centre of the endomorphism algebra as a Lie algebra is $A$ (?????), hence the above property effectively defines $\theta$. 
%    \item Can define $\bC$-algebra of differentials $Diff(A)$ as the subalgebra of $\End_\bC(A)$ generated by $\Der_\bC(A)$ and $A$ itself.
    \item Define a differential of order $\leq n$ to be an endo such that
        \[
            [ [ [ [\theta, a_1], a_2], \dots], a_n] = 0.
        \]
    \item Define space of differential operators $\Diff_\bC(A)$ to be all endos of the above form.
    \item Looks unnatural, but by the above $\theta \in \Der(A)$ is a differential of order $1$.
    \item Also, say, given $\theta_1, \theta_2 \in \Der(A)$ we have
        \begin{align*}
            [\theta_1\theta_2, f] &= \theta_1\theta_2f - f\theta_1\theta_2 \\
                &= \theta_1\theta_2 f - \theta_1 f \theta_2 + \theta_1 f \theta)2 - f\theta_1\theta_2 \\
                &= \theta_1[\theta_2, f] + [\theta_1, f]\theta_2
        \end{align*}
        so we see that it is fairly natural for an $n$th order derivative to go down an order in the Lie bracket.
    \item As a consequence of the above by induction, the order of the product of a diff of order $m$ and one of order $n$ is at most $m+n$.
    \item We naturally define a filtration by order on $Diff$, letting
        \[
            F_l \Diff_\bC(A) = \{\theta \in \Diff_\bC(A) \st order(\theta) \leq l\}
        \]
    \item By our inductive definition this is an increasing filtration of $\Diff_\bC(A)$ by $\bC$-modules.
    \item The filtration is compatible with the ring structure of $Diff_\bC(A)$, since
        \[
            F_m \circ F_n \subseteq F_{m+n}.
        \]
    \item It is fairly easy in this setting to show
        \begin{itemize}
            \item $F_0 Diff_\bC(A) = A$
            \item $F_1 Diff_\bC(A) = \Der_\bC(A) \oplus A$
            \item $[F_m, F_n] \subset F_{n+m-1}$.
        \end{itemize}
    \item We can then construct the graded ring from these filtrations, which is a commutative $A$-algebra.
    \item We now discuss an important example. To understand differentials on an affine alg. var, we look at them on an affine algebra, for which we look at them on a polynomial ring.
    \item Let $\partial_i$ be the standard derivations on the polynomial ring $\bC[x_1,\dots,x_n]$, it is fairly easy to show that they form a basis for $\Der_\bC(\bC[x_1,\dots,x_n])$ as a free $\bC[x_1,\dots,x_n]$ module.
    \item The differential operators then lie in
        \[
            \sum_{\alpha \in \bN^n} \bC[X] \partial_x^{\alpha}.
        \]
    \item $\Diff_\bC(\bC[x_1,\dots,x_n])$ is the $\bC$-algebra generated by $\{x_i, \partial_i\}_{1\leq i \leq n}$ such that
        \[
            [x_i, x_j] = 0, \qquad [\partial_i, \partial_j] = 0, \qquad [\partial_i, x_j] = \delta_{ij}.
        \]
\end{itemize}

\newpage

\begin{itemize}
    \item We have covered what the algebra of differential operators is over a $\bC$-algebra. We now describe differential operators in the sheaf setting, i.e. on a smooth algebraic variety. They will appear much the same, however we will need to do several annoying constructions in order to define them in a global sense.
    \item First, we define the constant sheaf $\bC_X$ over $X$.
    \item The constant sheaf assigns to each open set $U$ the set of locally constant functions from $U$ to $\bC$, i.e. continuous functions where $\bC$ is given the discrete topology.
        \[
            \bC_X(U) = \{f \in \Hom(U, \bC) \st \exists a\in \bC \st f(U) = a\}.
        \]
    \item If $X$ is a connected space, then $\bC_X(U) \isom \bC$ for every $U$.
    \item We defined algebraic varieties to be irreducible. Algebraic varieties may then be considered $\bC_X$-modules, in the sense of (Diagram here)
    \item We want to say that vector fields are $\bC$-linear endomorphisms of the structure sheaf satisfying the Leibniz rule.
    \item The endomorphism sheaf $\End_{\bC_X}(\cO_X)$ is the internal hom sheaf where open sets $U$ are sent to the $\bC$ algebra of sheaf homs
        \[
            \Hom_{\bC_X}(\cO_X|_U, \cO_X|_U).
        \]
    \item Being sheaf homs, each homomorphism induces $\bC$-algebra homomorphisms on every open subset of $U$, compatible with the restriction maps.
    \item A vector field is then defined as usual, $\theta \in \End_{\bC_X}(\cO_X)$ such that locally, for every $U$ we satisfy
        \[
            \theta(fg) = \theta(f)g + f\theta(g)
        \]
        where $\theta = \theta|_U = \theta(U)$.
    \item Note that for affine opens, this will become the twiddlificiation of $\Der_\bC(A)$, where $A$ is the underlying affine algebra.
    \item We define the tangent sheaf by sending $U$ to every $\theta(U)$. Note that by the above, it is a quasi-coherent $\cO_X$-module, and of course the trivial Lie Algebroid as mentioned in the last talk.
    \item In fact, the tangent sheaf is coherent since each $A$ is finitely generated and so as we saw above $\Der_{\bC}(A)$ is a free $A$-module.
    \item Just as we looked at $A$ as a subalgebra of $\End(A)$, we can look at $\cO_X$ as a subsheaf of $\End_{\bC_X}(\cO_X)$.
    \item A subsheaf assigns to each $U$ a sub collection of objects (preserving algebra structure) with the natural induced restriction maps.
    \item $D_X$ is defined to be the $\bC$-subalgebra of $\End_{\bC_X}(\cO_X)$ generated by $\cO_X$ and $\Theta_X$. What does this mean? intuitively, consider affine open $(U, \cO_X|_U)$. Then, consider the $A$-module of differentials $\Diff_\bC(A)$ on the associated algebra $A$. The module is the subalgebra of $\End_{\bC(A)}(A)$ generated by $\cO_X|_U$ and $\Der_\bC(A)$. We can then twiddlify this module to obtain $D_X|_U$. Patching this together gives us the sheaf of differentials.
    \item We can also give a local constructive definition of the sheaf of differentials, by saying for an affine open $U$, $D_X|_U$ is the free $\bC$-algebra generated by $\theta \in \Theta$ and $f \in \cO_X$ quotient by some expected relations: (relations)
    \item Take a local coordinate system on the tangent sheaf for a given affine open $U$. We can do this since we are smooth.
        \[
            \Theta_X|_U = \bigoplus_{1\leq i \leq n} \cO_X|_U \partial_i
        \]
    \item Similarly, the $D_X$ module can be written
        \[
            D_X|_U = \bigoplus_{\alpha \in \bN^n} \cO_X|_U \partial_x^\alpha.
        \]
    \item We can define the order of a differential operator to be the highest total power of $\partial$s. As before, we can then obtain an ascending filtration on order for $D_X|_U$:
        \[
            F_l D_X|_U \ceq \sum_{|\alpha| < l} \cO_X|_U \partial_x^\alpha
        \]
    \item We can generalise this to an order filtration on $D_X$ by saying $P\in (F_lD_X)(V)$ holds if $\rho_{UV}(P) \in F_lD_X|_U$ for some affine open $U$.
    \item We then obtain that $F_l D_X$ forms an exhaustive increasing filtration of $D_X$, of locally free modules over $\cO_X$. We also obtain properties we showed for the algebraic case:
        \begin{itemize}
            \item $F_0 D_X = \cO_X$, $(F_l)(F_m) = F_{l+m}D_X$
            \item $[F_l, F_m] \subseteq F_{l+m-1} D_X$.
        \end{itemize}
    \item The graded ring associated to this filtration is a sheaf of commutative rings finitiely generated over $\cO_X$.
\end{itemize}
We now have all the tools we need to define a $D$-module.
\begin{itemize}
    \item Technically, a $D_X$-module $M$ can be defined similarly to an $\cO_X$ module by simply saying that each $M(U)$ must be a $D_X(U)$ module such that the action commutes with the restriction maps. (diagram)
    \item However, we have a nicer characterisation, which says that a $D_X$-module is an $\cO_X$ module with an action from the tangent sheaf satisfying the properties of a flat connection.
    \item $\bC$-linear morphism 
        \[
            \vfunc{\nabla}{\Theta_X}{\End_\bC(M)}{\theta}{\nabla_\theta}
        \]
        satisfying
        \begin{enumerate}
            \item $\nabla_{f\theta}(s) = f\nabla_{\theta}(s)$ \\
            \item $\nabla_\theta(fs) = \theta(f)s = f\nabla_\theta(s)$ \\
            \item $\nabla_{[\theta_1, \theta_2]}(s) = [\nabla_{\theta_1}, \nabla_{\theta_2}](s)$
        \end{enumerate}
        module structure given by
        \[
            \theta s = \nabla_\theta(s)
        \]
    \item First two properties define a connection, last one defines flatness.
\end{itemize}
\end{document}
