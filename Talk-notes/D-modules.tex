\documentclass[a4paper]{article}

\usepackage[notes]{fancymath}
\usepackage{fancycom}

\aname{Adam Walsh}
\stitle{D-modules Talk notes}

\begin{document}
\maketitle
We aim to define the notion of a $D$-module, and see how we can represent the classical notion of a differential equation as a $D$-module.

Consider a commutative algebra $A$ over $\bC$, and the ring of $\bC$-linear endomorphisms $\End_\bC(A)$.\\

The elements of $A$ may be considered as acting on $A$ by multiplication, and we can identify $\End_A(A)$ with $A$.\\

A $\bC$-derivation of $A$ is a $\bC$-linear endomorphism $\theta$ of $A$ satisfying the the Leibniz rule:
\[
    \theta(fg) = \theta(f)g + f\theta(g).
\]
Note that
\[
    [\theta,f](g) = (\theta f)(g) - (f\theta)(g) = \theta(fg) - f\theta(g) = (\theta(f))(g).
\]
Hence the action of a derivation on an element of $A$ may be written $[\theta, f]$ where $f$ is the corresponding endomorphism.\\
Since $[\theta, f] = \theta(f) \in A$, and $A$ is commutative, we see that $[[\theta,f ], g] = 0$ holds for any derivations $\theta$ and sections $f,g$. Later we come back to this and see how it encodes the notion that $\theta$ is a ``first order'' derivation.

We saw schemes in the last talk. An algebraic variety over $\bC$ is a finite-type reduced integral scheme (some things redundant?) where the structure sheaf is a sheaf of $\bC$-algebras and the sections may be considered $\bC$-valued functions.


A vector field on a manifold is simply a section of the tangent bundle on that manifold. That is, the vector field $\theta$ assigns to each point $p\in M$, a tangent vector $\theta_p$ in the tangent space over $p$. The tangent vector may be considered to act as a $\bC$-linear endomorphism on $C_\infty(M)$ as a ``derivation'', i.e. satisfying the Leibniz rule.

Consider instead an algebraic variety $(X, \cO_X)$. We define vector fields on $X$ by this property. Denote the sheaf of $\bC$-linear endomorphisms of $\cO_X$ by $\End_{\bC_X}(\cO_X)$ (these commute with the restriction maps?). A section of this sheaf is locally just a $\bC$-linear endomorphism of algebras. A vector field on $X$ $\theta \in \End_{\bC_X}(\cO_X)$ is required to locally satisfy the Leibniz property, i.e. for each open $U$ in $X$, the restriction $\theta|_U$ satisfies
\[
    \theta|_U(fg) =\theta|_U(f) g + f \theta|_U(g)
\]
for all sections $f,g \in \cO_X(U)$.
(TBD: this notation is imprecise, vector fields should be defined locally)

Just in the case of manifolds and the tangent sheaf, these vector fields form a sheaf $\Theta$ on $X$, by assigning each open set $U$ to the collection of vector fields restricted to $U$. In fact, $\Theta$ has $\cO_X$-module structure, since we can act a section $f\in \cO_X(U)$ on a vector field $\theta \in \Theta(U)$ by left multiplication. 

Consider an affine open $U$ in $X$. Then saying that the vector fields $\theta \in \Theta(U)$ act $\bC$-linerarly on $\cO_X(U)$ and satisfy the Leibniz rule, is to say that it acts as a derivation on the $\bC$-algebra $\cO_X(U)$. From this we see that
\[
    \Theta|_{U} \isom \tilde{\Der}_\bC(A = \cO(U)).
\]
Hence, on every affine open, $\Theta$ is isomorphic to the twiddlification of some $\cO_X$-module, i.e. $\Theta$ is quasi-coherent. In fact, the $A$-module $\Der_\bC(A)$ is finitely generated.

The global sections of an affine algebraic variety over $\bC$ form an ``affine $\bC$-algebra'', which is in particular isomorphic to the polynomial ring $\bC[x_1,\dots,x_n]$ for some $n$ quotient by some ideal $I$. The derivations of the polynomial ring may be written
\[
    \Der_{\bC}(\bC[x_1,..,x_n]) = \bigoplus_{i\leq n} \bC[x_1,\dots,x_n] \frac{\partial}{\partial x_i}
\]
where $\frac{\partial}{\partial x_i}$ is the derivation on the polynomial ring such that $\pp{x_i} x_j = \delta_{ij}$ and $\pp{x_i}\pp{x_j} = \pp{x_j}\pp{x_i}$.\\
The derivations on the affine algebra correspond to that subset of these which fix $I$, and hence form a finitely generated $A$-module.\\
We thus see that the tangent sheaf is a coherent $\cO_X$-module.\\

We call an algebraic variety smooth if the stalks are \emph{regular} local rings, i.e. the minimal number of generators for the maximal ideal is equal to the Krull dimension.\\
This gives us some nice properties, for instance the stalks become unique factorisation domains. More relevantly, we can define the notion of dimension for a smooth affine variety, which turns out to be the number of variables in the polynomial ring above.\\
We now define the notion of a local coordinate system. At every point $p$ on $X$, there is an affine open neighborhood $U$ of $p$, sections $x_i \in \cO_X(U)$ and corresponding vector fields $\ptl_i \in \Theta(U)$ such that $\Theta(U)$ is generated over $\cO_X(U)$ with basis $\ptl_i$, $\ptl_i(x_j) = \delta_{ij}$ and the $\ptl_i$ commute. That is,
\[
    \begin{cases}
        [\ptl_i, \ptl_j] = 0, \ptl_{i}(x_i) = \delta_{ij} &\qquad 1\leq i,j \leq n\\
        \Theta(U) = \bigoplus_{1\leq i \leq n} \cO_U \ptl_{i}.
    \end{cases}
\]
The collection $\{x_i, \ptl_i\}$ is known as a local coordinate system. In the manifold case, the $x_i$ are precisely coordinate functions on a chart, and $\ptl_i$ are the tangent vectors defined by those coordinate functions.



We now have all the tools we require to define the sheaf of differentials.

Let $X$ as usual be a smooth algebraic variety with structure sheaf $\cO_X$, and let $\Theta_X$ denote the tangent sheaf on $X$. Sections in $\cO_X$ may be considered as $\bC$-valued functions, and by left multiplication we may consider them locally $\bC$-linear endomorphisms of $\cO_X$. The sections of $\Theta_X$ are vector fields, by definition they are locally $\bC$-linear endomorphisms of $\cO_X$. Accordingly, we may generate the $\End_{C_X}(\cO_X)$ subalgebra consisting locally of all polynomials of sections in $\cO_X$ and derivations in $\Theta_X$. These polynomials form the sheaf $D_X$, known as the sheaf of differentials.


Take any affine open $U$ of $X$, then $D_X(U)$ is the $\bC$-algebra generated by the $\bC$-algebra $\cO_X(U)$ and the $\cO_X(U)$-module $\Theta_X(U)$ preserving the algebra structure on $\cO_X(U)$, the Lie algebra structure on $\Theta_X(U)$ and the actions of each on the other. That is, $D_X(U)$ is the free algebra generated by the symbols $f\in \cO_X(U)$ and $\theta \in \Theta_X(U)$ quotient by the relations:
\[
    \begin{cases}
        \tilde{f_1} + \tilde{f_2} = \tilde{f_1 + f_2}, \quad \tilde{f_1f_2} = \tilde{f_1}\tilde{f_2}\\
        \tilde{\theta_1 + \theta_2} = \tilde{\theta_1} + \tilde{\theta_2}, \quad \widetilde{[\theta_1, \theta_2]} = [\tilde{\theta_1}, \tilde{\theta_2}] \\
        \widetilde{f\theta} = \tilde{f}\tilde{\theta}, \quad \widetilde{\theta(f)} = [\tilde{\theta}, \tilde{f}]
    \end{cases}
\]

Consider a local coordinate system $\{x_i, \ptl_i\}$ for the tangent sheaf $\Theta_X$ on an affine open $U$. $D_X(U)$ is formed of all polynomials of derivations and sections. Using standard multindex notation, we can write
\[
    D_X(U) = \bigoplus_{\alpha \in \bN^\alpha} \cO_U \ptl^\alpha_x\qquad \ptl^\alpha_x = \ptl^{\alpha_1}_{1}\ptl^{\alpha^2}_{2}\dots\ptl^{\alpha_n}_{n}.
\]
Each differential operator on $U$ can then be written as a polynomial
\[
    P = \sum_{\alpha \in \bN^\alpha} a_\alpha(x) \ptl^\alpha_x
\]
with coefficients $a_\alpha(x) \in \cO_X(U)$ (TBD: are these coefficients linear functions). We define the total symbol of a differential operator as above to be the corresponding polynomial
\[
    \sigma(P) = \sum_{\alpha \in \bN^\alpha} a_\alpha(x) \psi^\alpha_x \in \bC[x_1,\dots,x_n,\psi_1,\dots,\psi_n].
\]
The order of a differential operator is the maximum $|\alpha|$ for which $a_\alpha$ is non-zero. This order is also the minimum $l$ for which
\[
    [ [\dots [[ P, a_0], a_1], \dots], a_l ] = 0.
\]
We may define a filtration by order of $D_X(U)$ as follows:
\[
    F_l D_X(U) = \bigoplus_{|\alpha| < l} \cO_U \ptl^\alpha_x.
\]
This can be extended to an arbitrary open subset $V$ by letting $F_lD_X(V)$ consist of those differentials of $V$ which restrict into $F_lD_X(U)$ for all open affines $U$.

\end{document}

