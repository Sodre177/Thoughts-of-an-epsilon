\documentclass[a4paper]{amsart}

\usepackage{fancycom}

\title{Questions}
\author{Adam Walsh}

\begin{document}
\maketitle

\section{Important}
\section{Major}
\section{Minor}
\begin{itemize}
    \item Why is a local ring called local? 
        \begin{itemize}
            \item (My guess is that this name is motivated by what is written on page 40 - Local properties)
        \end{itemize}
    \item What about a residue field?
    \item Does a prime element make sense outside of a unique factorisation domain? If it does, will the ideal generated by the prime element itself be prime.  
    \begin{itemize}
        \item If the definition runs something like an element with no non-unit divisors, then no, since $x=ap=bq$ with $a,p,b,q$ prime might occur.
        \item (I think so, I think outside of a UFD it just loses uniqueness)
    \end{itemize}
    \item Why do prime ideals containing an ideal correspond to prime ideals in the quotient ring of that ideal. 
    \begin{itemize}
        \item 
            %Irrelevant: We are considering $V(\mathfrak{q})\cong \text{Spec}(A/\mathfrak{q})$ I think. Relevant: If $\mathfrak{p}\subset V(\mathfrak{q})$, then $\mathfrak{p}\supset \mathfrak{q}$. Now, consider $\varphi:A\to A/\mathfrak{q}$. Then, we know that $\varphi(\mathfrak{p})\subset A/\mathfrak{q}$ is an ideal, but do not know if it is prime. Consider $0\ne (ab+\mathfrak{q})\in \varphi(\mathfrak{p})$. Since $A/\mathfrak{q}$ is an integral domain, this implies that $ab\not\in\mathfrak{q}$. Since $\mathfrak{q}$ is an ideal, it follows that neither $a$ nor $b$ are in $\mathfrak{q}$, but since $ab\in\mathfrak{p}$, either $a$ or $b$ \textit{is} in $\mathfrak{p}$, and hence mapped to nonzero elements of $A/\mathfrak{q}$ by $\varphi$. It follows that $\varphi(\mathfrak{p})$ contains $a$ or (and?) $b$.
            %\\
            %\\
            We want to show that for an ideal $\mathfrak{q}\subset A$ (that is not necessarily prime) and $\mathfrak{p}\in V(\mathfrak{q})$, that $\varphi(\mathfrak{p})$ is prime in $A/\mathfrak{q}$.
            \\
            \\
            (Recall that $V(\mathfrak{p})=\{\mathfrak{r}\in\text{Spec}(A)\mid \mathfrak{r}\supset \mathfrak{p}\}$.)
            \\
            \\
            Let $\mathfrak{p}\in V(\mathfrak{q})$ and let $\varphi:A\to A/\mathfrak{q}$. We know that $\varphi(\mathfrak{p})$ is an ideal, we simply need to check if it is prime. Consider $(ab+\mathfrak{q})\in \varphi(\mathfrak{p})$. This means that for some $c\in \mathfrak{p}$, $\varphi(c)=ab+\mathfrak{q}$, i.e. $c=ab+d$ for some $d\in \mathfrak{q}$. But $\mathfrak{q}\subset \mathfrak{p}$, and these are abelian groups under addition, and hence $-d\in\mathfrak{p}$. This implies that $ab+d-d\in\mathfrak{p}$ and hence $ab\in\mathfrak{p}$. Since $\mathfrak{p}$ is prime, $a\in\mathfrak{p}$ or $b\in \mathfrak{p}$, and hence $a+\mathfrak{q}$ or $b+\mathfrak{q}$ is in $\varphi(\mathfrak{p})$ as desired.
            \\
            \\
            Didn't use this in the end, but its fun to think about: $A-A^\times = \bigcup_{\mathfrak{p}\in\text{Spec}(A)} \mathfrak{p}$.
    \end{itemize}
    \item What gives a free functor from the category of abelian groups to the category of commutative rings?
    \item The natural complex structure on $\bC$ is the one represented by the global identity chart. Are all atlases definable on $\bC$ analytically equivalent to this identity chart? i.e. is it possible for a chart to be defined on $\bC$ which is not holomorphically compatible to the global identity chart.
    \item As part of the above question, what do the homeomorphisms from $\bC$ to itself which aren't holomorphic look like, if there are any? (The answer will be the same as the answer to the above question).
    \item How does one formally prove (category theoretically) that the action of an element of a ring on each term of a product of modules being invertible means that the action on the product is invertible?
\end{itemize}

\end{document}

